\chapter{Conclusões}
\label{chap:final_remarks}

\noindent Neste projeto foi feito um trabalho de exploração de um problema recente e aplicação de métodos vistos ao longo da disciplina de Programação Linear Inteira. Foi possível ainda experimentar uma linguagem de programação nova e voltada à otimização e também experimentar heurísticas diferentes.

Para o método de fixação de variáveis ainda há espaço para explorar mais tipos de fixações e também de operadores, além disso, buscar métodos mais eficazes para viabilizar soluções na parte de relaxação Lagrangiana também é uma alternativa. Mesmo que durante todos os testes a impressão mais forte que ficou é que a Relaxação Lagrangiana talvez não seja um método muito adequado para este problema, ou que pelo menos ainda existem mais possibilidades de dualizações a serem exploradas. Por último, talvez também seja interessante aliar as propostas atuais ao \emph{Concorde TSP solver}\cite{ConcordeTSPSolver}, que é reconhecido na literatura como extremamente eficiente para o TSP -- mesmo que seja para o TSP simétrico, é possível adaptar o processamento das entradas para encaixar com as necessidades dele. 

Com as implementações realizadas e após os experimentos, nota-se que ainda há muito espaço para melhora, tanto no uso de recursos quanto na parametrização dos algoritmos desenvolvidos. Apesar disso, é importante destacar o nível de discussão produzido ao atacar um problema novo. Adicionando uma nota pessoal: tive a oportunidade de discutir extensamente com os colegas sobre as abordagens, as limitações e dificuldades. Com essa rede de apoio foi possível, em um prazo relativamente curto, superar barreiras de conhecimento tanto teórico quanto técnico.